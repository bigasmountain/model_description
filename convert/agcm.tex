\documentclass[12pt,a4paper,onecolumn]{jarticle}
%
\usepackage{amsmath,amssymb}
\usepackage{bm}
\usepackage{graphicx}
\usepackage{ascmac}
\usepackage[dvipdfm]{hyperref}
%
\title{\Huge CCSR/NIES AGCM の解説}
\author{}
\date{\today}
\pagenumbering{arabic}
%
%
\begin{document}
%
	\maketitle
	\tableofcontents
	\clearpage
	%
	% ccsr/nies agcm の特徴の list
	\include{tex_en/summary}
	% agcm の概念と構造
	\include{tex_en/a-intro}
	% 基本設定
	\include{tex_en/a-setup}
	% 基本方程式
	\include{tex_en/d-basic}
	% 力学:鉛直離散化
	\include{tex_en/d-vert}
	% 力学:水平離散化
	\include{tex_en/d-hori}
	% 力学:時間離散化
	\include{tex_en/d-time}
	% 力学:拡散項等
	\include{tex_en/d-diff}
	% 力学:まとめ
	\include{tex_en/d-summ}
	% 物理過程:イントロ
	\include{tex_en/p-intro}
	% 物理過程:積雲対流
	\include{tex_en/p-cum}
	% 物理過程:大規模凝結
	\include{tex_en/p-lsc}
	% 物理過程:放射
	\include{tex_en/p-rad}
	% 物理過程:拡散フラックス
	\include{tex_en/p-dif}
	% 物理過程:地表フラックス
	\include{tex_en/p-sflx}
	% 物理過程:地表モデル
	\include{tex_en/p-sfc}
	% 物理過程:implicit 解法
	\include{tex_en/p-solv}
	% 物理過程:重力波抵抗
	\include{tex_en/p-grav}
	% 物理過程:対流調節
	\include{tex_en/p-adj}
	% 参考文献
	\include{tex_en/referenc}
	%
	%
\end{document}
