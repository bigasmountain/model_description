\hypertarget{soil-submodel-matgnd}{%
\section{Soil Submodel MATGND}\label{soil-submodel-matgnd}}

Calculating soil temperature, soil moisture and frozen ground.

\hypertarget{calculating-heat-transfer-in-soil.}{%
\subsection{Calculating heat transfer in
soil.}\label{calculating-heat-transfer-in-soil.}}

\hypertarget{the-heat-transfer-equation-in-soil}{%
\subsubsection{The Heat Transfer Equation in
Soil}\label{the-heat-transfer-equation-in-soil}}

The prediction equation for soil temperature due to heat transfer in the
soil is as follows.

\begin{eqnarray}
C_{g(k)} \frac{T_{g(k)}^* - T_{g(k)}^{\tau}}{\Delta t_L} = F_{g(k+1/2)} - F_{g(k-1/2)}
\qquad (k=1,\ldots,K_{g})
\end{eqnarray}

\(C_{g(k)}\) is the heat capacity of the soil and is defined by

\begin{eqnarray}
 C_{g(k)} = ( c_{g(k)} + \rho_w c_{pw} w_{(k)} ) \Delta z_{g(k)}
\end{eqnarray}

\(c_{g(k)}\) is the specific heat of the soil and is given as a
parameter for each soil type. \(c_{pw}\) is the specific heat of water
and \(w_{(k)}\) is the soil water content (volume moisture content).
\(\Delta z_{g(k)}\) is the thickness of the \(k\) layer of soil. Thus,
when the heat capacity of the soil is included in the heat capacity of
the soil, the energy is not conserved unless the heat transport due to
soil moisture transfer is taken into account. Since the heat transport
associated with soil moisture transfer is not considered in MATGND, we
are now discussing its introduction. However, it should be noted that
the energy conservation is somehow broken unless the heat capacity of
water vapor and precipitation in the atmosphere is taken into account.

The heat transfer flux \(F_{g}\) is given by

\begin{eqnarray}
 F_{g(k+1/2)} =
\left\{
\begin{array}{ll}
F_{g(1/2)} - \Delta F_{conv}^* - \Delta F_{c,conv}^*
 (k=0)\\
\displaystyle{
k_{g(k+1/2)} \frac{T_{g(k+1)} - T_{g(k)}}{\Delta z_{g(k+1/2)}}
}
 (k=1,\ldots,K_{g}-1) \\
\displaystyle{
0
}
 (k=K_{g})
\end{array}
\right.
\end{eqnarray}

where \(k_{g(k+1/2)}\) is the thermal conductivity of the soil and is
given as follows

\begin{eqnarray}
 k_{g(k+1/2)} = k_{g0(k+1/2)} [ 1 + f_{kg} \tanh( w_{(k)}/ w_{kg} ) ]
\end{eqnarray}

\(k_{g0(k+1/2)}\) is the thermal conductivity of the soil when the soil
moisture content is \(0\), and \(f_{kg}=6\) and \(w_{kg}=0.25\) are
constants.

\(\Delta z_{g(k+1/2)}\) is the thickness between the temperature
definition point of the first \(k\) layer and the soil temperature
definition point of the \(k+1\) layer (for \(k=0\), it is the thickness
between the temperature definition point of the first layer and the top
of the soil, and for \(k=K_g\), it is the thickness between the
temperature definition point of the lowest layer and the bottom of the
soil).

In (3), the boundary condition (\(F_{g(1/2)}\)) is obtained by adding
the energy convergence at the bottom edge of the snowpack (including the
heat flux at the bottom edge of the snowpack) and the assignment of the
energy correction term to the snow-free surface due to the phase change
of water content in the canopy. The fluxes are given. The fluxes are
positive upward and are negative when adding the convergence amount. The
boundary condition \(F_{g(K_g+1/2)}\) at the lower edge of the soil is
assumed to be zero flux.

\hypertarget{solving-the-heat-transfer-equation.}{%
\subsubsection{Solving the heat transfer
equation.}\label{solving-the-heat-transfer-equation.}}

These equations are solved in terms of soil temperature from the first
to the lowest layer using the implicit method. In other words, for
\(k=1,\ldots,K_g-1\), the heat transfer fluxes are estimated by

\begin{eqnarray}
  F_{g(k+1/2)}^{*} = F_{g(k+1/2)}^{\tau}
+\frac{\partial {F}_{g(k+1/2)}}{\partial T_{g(k)}}
 \Delta T_{g(k)}
+\frac{\partial {F}_{g(k+1/2)}}{\partial T_{g(k+1)}}
 \Delta T_{g(k+1)}
\end{eqnarray}

\begin{eqnarray}
  F_{g(k+1/2)}^{\tau} =
\frac{k_{g(k+1/2)}}{\Delta z_{g(k+1/2)}}(T_{g(k+1)}^{\tau} - T_{g(k)}^{\tau})
\end{eqnarray}

\begin{eqnarray}
 \frac{\partial {F}_{g(k+1/2)}}{\partial T_{g(k)}} =
- \frac{k_{g(k+1/2)}}{\Delta z_{g(k+1/2)}}
\end{eqnarray}

\begin{eqnarray}
 \frac{\partial {F}_{g(k+1/2)}}{\partial T_{g(k+1)}} =
\frac{k_{g(k+1/2)}}{\Delta z_{g(k+1/2)}}
\end{eqnarray}

and then add (1) to

\begin{eqnarray}
C_{g(k)} \frac{\Delta T_{g(k)}}{\Delta t_L}
= F_{g(k+1/2)}^* - {F}_{g(k-1/2)}^*  \\
= {F}_{g(k+1/2)}^{\tau}
+\frac{\partial F_{g(k+1/2)}}{\partial T_{g(k)}}
 \Delta T_{g(k)}
+\frac{\partial F_{g(k+1/2)}}{\partial T_{g(k+1)}}
 \Delta T_{g(k+1)}  \\
- F_{g(k-1/2)}^{\tau}
-\frac{\partial F_{g(k-1/2)}}{\partial T_{g(k-1)}}
 \Delta T_{g(k-1)}
-\frac{\partial F_{g(k-1/2)}}{\partial T_{g(k-1)}}
 \Delta T_{g(k)}
\end{eqnarray}

and solved by the LU decomposition method as a series of \(K_{g}\)
equations for \(\Delta T_{g(k)}\ (k=1,\ldots,K_{g})\). Note that the
fluxes at the top and bottom of the soil are fixed as boundary
conditions.

\begin{eqnarray}
 T_{g(k)}^* = T_{g(k)}^{\tau} + \Delta T_{g(k)}
\end{eqnarray}

After correction for the phase change of soil moisture content, which
will be described later, the soil temperature is completely updated.

\hypertarget{calculation-of-soil-moisture-transfer}{%
\subsection{Calculation of soil moisture
transfer}\label{calculation-of-soil-moisture-transfer}}

\hypertarget{equation-for-soil-moisture-transfer}{%
\subsubsection{Equation for Soil Moisture
Transfer}\label{equation-for-soil-moisture-transfer}}

The equation for soil moisture transfer (Richards' equation) is given by

\begin{eqnarray}
\rho_w \frac{w_{(k)}^{\tau+1} - w_{(k)}^{\tau}}{\Delta t_L} =
\frac{F_{w(k+1/2)} - F_{w(k-1/2)}}{\Delta z_{g(k)}} + S_{w(k)}
\qquad (k=1,\ldots,K_{g})
\end{eqnarray}

Soil moisture flux \(F_{w}\) is given by

\begin{eqnarray}
 F_{w(k+1/2)} =
\left\{
\begin{array}{ll}
Pr^{***} - Et_{(1,1)}
 (k=0)\\
\displaystyle{
K_{(k+1/2)} \left(\frac{\psi_{(k+1)} - \psi_{(k)}}{\Delta z_{g(k+1/2)}} - 1 \right)
}
 (k=1,\ldots,K_{g}-1) \\
\displaystyle{
0
}
 (k=K_{g})
\end{array}
\right.
\end{eqnarray}

where \(K_{(k+1/2)}\) is the soil permeability coefficient based on
Clapp and Hornberger (1978) and is given as follows

\begin{eqnarray}
 K_{(k+1/2)} = K_{s(k+1/2)} (\max(W_{(k)},W_{(k+1)}))^{2b(k)+3} f_i
\end{eqnarray}

\(K_{s(k+1/2)}\) is the saturated hydraulic conductivity and \(b_{(k)}\)
is the index of moisture potential curve as an external parameter for
each soil type. \(W_{(k)}\) is the saturation degree excluding freezing
soil moisture and is given by

\begin{eqnarray}
 W_{(k)} = \frac{w_{(k)}-w_{i(k)}}{w_{sat(k)}-w_{i(k)}}
\end{eqnarray}

\(w_{sat(k)}\) is the porosity of soil, which is also given as a
parameter for each soil type. \(f_i\) is the parameter that indicates
the suppression of soil moisture migration due to the presence of frozen
soil, and is currently given as follows.

\begin{eqnarray}
 f_i = \left(1- W_{i(k)}\right)
       \left(1- W_{i(k+1)}\right)
\end{eqnarray}

This parameter is \(W_{i(k)} = w_{i(k)}/(w_{sat(k)}-w_{i(k)})\).

\(\psi\) is the soil moisture potential given by Clapp and Hornberger as
follows

\begin{eqnarray}
 \psi_{(k)} = \psi_{s(k)} W_{(k)}^{-b(k)}
\end{eqnarray}

\(\psi_{s(k)}\) is given as an external parameter for each soil type.

In (11), \(S_{w(k)}\) is the source term and, considering the absorption
and runoff by roots, is given by

\begin{eqnarray}
 S_{w(k)} = - F_{root(k)} - Ro_{(k)}
\end{eqnarray}

In (12), the boundary condition \(F_{w(1/2)}\) is the difference between
the water flux (\(P^{***}\)) and the evaporation flux (\(Et_{(1,1)}\))
at the top of the soil through the runoff process. Apart from this, the
sublimation flux is subtracted from the frozen soil moisture in the
first layer prior to the calculation of soil moisture transfer.

\begin{eqnarray}
 w_{i(k)}^{\tau} = w_{i(k)}^{\tau} - Et_{(2,1)} \Delta t_L /(\rho \Delta z_{g(1)})\\
 w_{(k)}^{\tau} = w_{(k)}^{\tau} - Et_{(2,1)} \Delta t_L /(\rho \Delta z_{g(1)})
\end{eqnarray}

\hypertarget{solving-the-soil-moisture-transfer-equation}{%
\subsubsection{Solving the soil moisture transfer
equation}\label{solving-the-soil-moisture-transfer-equation}}

These equations are solved from the first to the lowest layer using the
implicit method. In other words, for \(k=1,\ldots,K_g-1\), the soil
moisture fluxes are estimated by

\begin{eqnarray}
  F_{w(k+1/2)}^{\tau+1} = F_{w(k+1/2)}^{\tau}
+\frac{\partial {F}_{w(k+1/2)}}{\partial w_{(k)}}
 \Delta w_{(k)}
+\frac{\partial {F}_{w(k+1/2)}}{\partial w_{(k+1)}}
 \Delta w_{(k+1)}
\end{eqnarray}

\begin{eqnarray}
  F_{w(k+1/2)}^{\tau} =
K_{(k+1/2)} \left(\frac{\psi_{(k+1)}^{\tau} - \psi_{(k)}^{\tau}}{\Delta z_{g(k+1/2)}} - 1 \right)
\end{eqnarray}

\begin{eqnarray}
 \frac{\partial {F}_{w(k+1/2)}}{\partial w_{(k)}} =
- \frac{K_{(k+1/2)}}{\Delta z_{g(k+1/2)}}
\left[
-b_{(k)} \frac{\psi_{s(k)}}{w_{sat(k)}-w_{i(k)}}W_{(k)}^{-b(k)-1}
\right]
\end{eqnarray}

\begin{eqnarray}
 \frac{\partial {F}_{w(k+1/2)}}{\partial w_{(k+1)}} =
 \frac{K_{(k+1/2)}}{\Delta z_{g(k+1/2)}}
\left[
-b_{(k)} \frac{\psi_{s(k+1)}}{w_{sat(k+1)}-w_{i(k+1)}}W_{(k+1)}^{-b(k)-1}
\right]
\end{eqnarray}

and (11) as

\begin{eqnarray}
\rho_w \Delta z_{g(k)} \frac{\Delta w_{(k)}}{\Delta t_L}
= F_{w(k+1/2)}^{\tau+1} - {F}_{w(k-1/2)}^{\tau+1} + S_{w(k)} \Delta z_{g(k)}  \\
= {F}_{w(k+1/2)}^{\tau}
+\frac{\partial F_{w(k+1/2)}}{\partial w_{(k)}}
 \Delta w_{(k)}
+\frac{\partial F_{w(k+1/2)}}{\partial w_{(k+1)}}
 \Delta w_{(k+1)}  \\
- F_{w(k-1/2)}^{\tau}
-\frac{\partial F_{w(k-1/2)}}{\partial w_{(k-1)}}
 \Delta w_{(k-1)}
-\frac{\partial F_{w(k-1/2)}}{\partial w_{(k-1)}}
 \Delta w_{(k)} + S_{w(k)} \Delta z_{g(k)}
\end{eqnarray}

The equations are treated as follows for
\(\Delta T_{g(k)}\ (k=1,\ldots,K_{g})\), and are solved by the LU
decomposition method as a series of \(K_{g}\) equations for
\(\Delta T_{g(k)}\ (k=1,\ldots,K_{g})\). Note that the fluxes at the top
and bottom of the soil are fixed as boundary conditions.

\begin{eqnarray}
 w_{(k)}^{\tau+1} = w_{(k)}^{\tau} + \Delta w_{(k)}
\end{eqnarray}

The soil moisture content is updated using the LU decomposition method.

If this calculation results in supersaturation of soil moisture, the
supersaturation is removed by vertical adjustment. The supersaturation
is not considered as runoff because this supersaturation is artificial
and is caused by the solution of the vertical soil moisture transfer
without information about the saturation. First, supersaturated soil
moisture is applied from the second soil layer downward. Then, from the
lowermost layer of soil to the uppermost layer, the supersaturated soil
moisture is added to the uppermost layer. This operation results in the
formation of a saturated layer near the bottom of the soil when soil
moisture is sufficiently high to define the groundwater level (with a
certain amount of water content of the ground surface in the vicinity of
the lowest layer of soil (with a certain amount of water content of the
ground surface in the vicinity of the lowest level of soil).

\hypertarget{phase-changes-in-soil-moisture.}{%
\subsection{Phase changes in soil
moisture.}\label{phase-changes-in-soil-moisture.}}

As a result of the heat conduction in the soil, the phase change of soil
moisture content is calculated when the temperature of the layer with
liquid moisture is below \(T_{melt} = 0^{\circ}\) C or when the
temperature of the layer with solid moisture is above \(T_{melt}\).
Assuming that the freezing rate of soil moisture in the \(k\)st layer is
set to \(\Delta w_{i(k)}\),

In case of \(T_{g(k)}^*<T_{melt}\) and
\(w_{(k)}^{\tau+1}-w_{i(k)}^{\tau}>0\) (frozen)

\begin{eqnarray}
\Delta w_{i(k)} = \min\left(
\frac{C_{g(k)}(T_{melt}-T_{g(k)}^*)}{l_m \rho_w \Delta z_{g(k)}}, \
w_{(k)}^{\tau+1}-w_{i(k)}^{\tau}
\right)
\end{eqnarray}

In case of \(T_{g(k)}^*>T_{melt}\) and \(w_{i(k)}^{\tau}>0\) (melting)

\begin{eqnarray}
\Delta w_{i(k)} = \max\left(
\frac{C_{g(k)}(T_{melt}-T_{g(k)}^*)}{l_m \rho_w \Delta z_{g(k)}}, \
-w_{i(k)}^{\tau}
\right)
\end{eqnarray}

Update soil freezing moisture and soil temperature as follows.

\begin{eqnarray}
w_{i(k)}^{\tau+1} = w_{i(k)}^{\tau} + \Delta w_{i(k)} \\
T_{g(k)}^{\tau+1} = T_{g(k)}^* + l_m \rho_w \Delta z_{g(k)} \Delta w_{i(k)} / C_{g(k)}
\end{eqnarray}

\hypertarget{ice-sheet-process.}{%
\subsubsection{Ice sheet process.}\label{ice-sheet-process.}}

If land cover type is ice-cover, return to \(T_{melt}\) when soil
temperature exceeds \(T_{melt}\).

\begin{eqnarray}
 T_{g(k)}^{\tau+1} = \min( T_{g(k)}^*, \ T_{melt} )
\end{eqnarray}

The rate of change in the amount of ice cover, \(F_{ice}\), is diagnosed
as

\begin{eqnarray}
 F_{ice} = - Et_{(2,1)} - \frac{C_{g(k)}\max(T_{g(k)}^* - T_{melt},\ 0)}{l_m \Delta t_L}
\end{eqnarray}

Translated with www.DeepL.com/Translator (free version)
