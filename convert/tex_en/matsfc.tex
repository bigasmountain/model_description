\hypertarget{boundary-value-submodel-matbnd}{%
\section{Boundary Value Submodel
MATBND}\label{boundary-value-submodel-matbnd}}

\hypertarget{set-of-vegetation-shape-parameters.}{%
\subsection{Set of vegetation shape
parameters.}\label{set-of-vegetation-shape-parameters.}}

The leaf area index (LAI) and vegetation height are set as vegetation
shape parameters.

The LAI reads the seasonally varying horizontal distributions, and the
top and bottom canopy heights are determined by land use type as
external parameters. If there is snowfall, the LAI considers only the
vegetation above the snow depth and corrects the geometry parameters.

\begin{eqnarray}
 h = \max( h_0 - D_{Sn}, 0 ) \\
 h_B = \max( h_{B0} - D_{Sn}, 0 ) \\
 LAI = LAI_0 \frac{h-h_B}{h_0-h_{B0}}
\end{eqnarray}

Here, \(h\) is the height at the top of the canopy (vegetation height),
\(h_B\) is the height at the bottom of the canopy (dead height), \(LAI\)
is the leaf area index, and \(h_0\), \(h_{B0}\), and \(LAI_0\) are the
values without snow, respectively. \(D_{Sn}\) is the snow cover depth.
It is assumed that the LAI is uniformly distributed vertically between
the top and bottom edges of the canopy.

Afterwards, the average of snow-free and snow-covered surfaces weighted
by the snow area ratio (\(A_{Sn}\)) is calculated, but since the
snow-free surface and the snow-covered surface are calculated
separately, \(A_{Sn}\) requires either \(0\) (snow-free surface) or
\(1\) (snow-covered surface). Note that it contains either of the values
of the (surface), and mixing of values does not occur here (there are
several similar locations later).

\hypertarget{calculating-radiant-parameters.}{%
\subsection{Calculating radiant
parameters.}\label{calculating-radiant-parameters.}}

Calculation of radiative parameters (albedo, vegetation permeability,
etc.).

\hypertarget{calculation-of-surface-forest-floor-albedo}{%
\subsubsection{Calculation of surface (forest floor)
albedo}\label{calculation-of-surface-forest-floor-albedo}}

The horizontal distribution of the ground (forest floor) albedo
\(\alpha_{0(b)}\ \ (b=1,2)\) is read as an external parameter.
\(b=1, 2\) represent the visible and near-infrared wavelengths,
respectively. The \(\alpha_{0(3)}\) value of the infrared surface albedo
(\(\alpha_{0(3)}\)) is set to a fixed value (horizontal distribution may
be prepared).

The dependence of the incident angle of albedo on ice and snow cover is
considered as a function of the angle of incidence as follows

\begin{eqnarray}
 \alpha_{0(d,b)} = \hat{\alpha}_{0(b)} + ( 1 - \hat{\alpha}_{0(b)} )
                         \cdot 0.4 ( 1 - \cos \phi_{in(d)} )^5
\end{eqnarray}

where \(b=1,2\) are wavelengths, \(d=1,2\) are direct and scattered,
respectively, and \(\hat{\alpha}_{0(b)}\) is an albedo value for a
direct angle of incidence of \(0\) (from directly above).
\(\cos \psi_{in(d)}\) is the cosine of the incident angle of incidence
for direct and scattered light, respectively,

\begin{eqnarray}
 \cos\psi_{in(1)} = \cos\zeta, \ \ \
 \cos\psi_{in(2)} = \cos 50^{\circ}
\end{eqnarray}

We give the \(\zeta\) is the solar zenith angle.

Except for the ice and snow cover surfaces, the albedo of the ground
(forest floor) gives the same values for direct and diffuse light,
without considering the dependence on the zenith angle. In other words,
the results are as follows.

\begin{eqnarray}
 \alpha_{0(d,b)} = \alpha_{0(b)}\ \ \ (d=1,2;\ b=1,2)
\end{eqnarray}

For infrared wavelengths, we only need to consider the scattered light.
The infrared albedo gives a value that is independent of the zenith
angle for all surfaces.

\begin{eqnarray}
 \alpha_{0(2,3)} = \alpha_{0(3)}
\end{eqnarray}

\hypertarget{canopy-albedo-and-transmittance-calculations}{%
\subsubsection{Canopy albedo and transmittance
calculations}\label{canopy-albedo-and-transmittance-calculations}}

The calculation of albedo and transmittance of the canopy is based on
the radiation calculation in the canopy layer by Watanabe and Otani
(1995).

Considering a vertically uniform canopy and making some simplifying
assumptions, the transfer equation and boundary conditions for the
insolation within the canopy are expressed as follows

\begin{eqnarray}
 \frac{dS^{\downarrow}_d}{dL} = -F \sec\zeta S^{\downarrow}_d \\
 \frac{dS^{\downarrow}_r}{dL} = -F (1-t_{f(b)})d_f S^{\downarrow}_r
                                  +F t_{f(b)} \sec\zeta S^{\downarrow}_d
                                  +F r_{f(b)} d_f S^{\uparrow}_r \\
 \frac{dS^{\uparrow}_r}{dL}   =  F (1-t_{f(b)})d_f S^{\uparrow}_r
                                  -F r_{f(b)} ( d_f S^{\downarrow}_r
                                         + \sec\zeta S^{\downarrow}_d ) \\
 S^{\downarrow}_d(0) = S^{top}_d \\
 S^{\downarrow}_r(0) = S^{top}_r \\
 S^{\uparrow}_r(LAI) = \alpha_{0(1,b)}S^{\downarrow}_d(LAI)
                       + \alpha_{0(2,b)}S^{\downarrow}_r(LAI)
\end{eqnarray}

where \(S^{\downarrow}_d\) is direct downward light, \(S^{\uparrow}_r\)
and \(S^{\downarrow}_r\) are upward and downward scattered light,
respectively, \(L\) is the leaf area accumulated from the top of the
canopy downward, \(d_f\) is the scatter factor (\(=\sec 53^{\circ}\)),
\(r_{f(b)}\) and \(t_{f(b)}\) is the reflection coefficient and
transmission coefficient of the leaf surface (the same values are used
for scattered light and direct light), respectively, and \(F\) is a
factor indicating the orientation of the leaf relative to radiation. For
simplicity, we assume that the distribution of leaf orientation is
random (\(F=0.5\)).

These can be solved analytically and the solution is as follows.

\begin{eqnarray}
 S^{\downarrow}_d(L) = S^{top}_d \exp(-F\cdot L\cdot \sec\zeta) \\
 S^{\downarrow}_r(L) = C_1 e^{a L} + C_2 e^{-a L} + C_3 S^{\downarrow}_d(L) \\
 S^{\uparrow}_r(L)   = A_1 C_1 e^{a L} + A_2 C_2 e^{-a L} + C_4 S^{\downarrow}_d(L)
\end{eqnarray}

Here,

\begin{eqnarray}
   a = F d_f [(1-t_{f(b)})^2 - r_{f(b)}^2]^{1/2}  \\
 A_1 = \{ 1 - t_{f(b)} + [(1-t_{f(b)})^2 - r_{f(b)}^2]^{1/2}\} / r_{f(b)} \\
 A_2 = \{ 1 - t_{f(b)} - [(1-t_{f(b)})^2 - r_{f(b)}^2]^{1/2}\} / r_{f(b)} \\
 A_3 = (A_1 - \alpha_{0(2,b)}) e^{ a LAI }
        -(A_2 - \alpha_{0(2,b)}) e^{-a LAI } \\
 C_1 = \{ -(A_2 - \alpha_{0(2,b)}) e^{-a LAI} (S^{top}_r - C_3 S^{top}_d)
            +[C_3\alpha_{0(2,b)}+\alpha_{0(1,b)}-C_4]S^{\downarrow}_d(LAI)\} / A_3 \\
 C_2 = \{  (A_1 - \alpha_{0(2,b)}) e^{ a LAI} (S^{top}_r - C_3 S^{top}_d)
            -[C_3\alpha_{0(2,b)}+\alpha_{0(1,b)}-C_4]S^{\downarrow}_d(LAI)\} / A_3 \\
 C_3 = \frac{\sec\zeta[t_{f(b)}\sec\zeta + d_f t_{f(b)}(1-t_{f(b)}) + d_f r_{f(b)}^2]}
              {d_f^2[(1-t_{f(b)})^2-r_{f(b)}^2]-\sec^2\zeta} \\
 C_4 = \frac{r_{f(b)}(d_f - \sec\zeta)\sec\zeta}
              {d_f^2[(1-t_{f(b)})^2-r_{f(b)}^2]-\sec^2\zeta}
\end{eqnarray}

It is.

The Albedo \(\alpha_s\) seen at the top of the canopy,

\begin{eqnarray}
 S^{\uparrow}_r(0) = \alpha_{s(1,b)} S^{\downarrow}_d(0)
                   + \alpha_{s(2,b)} S^{\downarrow}_r(0)
\end{eqnarray}

So,

\begin{eqnarray}
 \alpha_{s(2,b)} = \{ A_2 ( A_1 - \alpha_{0(2,b)}) e^{ a LAI }
                      - A_1 ( A_2 - \alpha_{0(2,b)}) e^{-a LAI }
                   \} / A_3 \\
 \alpha_{s(1,b)} = - C_3 \alpha_{s(2,b)} + C_4
                  + ( A_1 - A_2 ) ( C_3 \alpha_{0(2,b)} + \alpha_{0(1,b)} -C_4)
                  e^{- F\cdot LAI\cdot \sec\zeta} / A_3
\end{eqnarray}

Get.

The transmission coefficient of the canopy (\({\mathcal{T}}_c\)), or
more precisely, the percentage of the incident light absorbed by the
forest floor at the top of the canopy, is

\begin{eqnarray}
 S^{\downarrow}_d(LAI) + S^{\downarrow}_r(LAI) - S^{\uparrow}_r(LAI)
= {\mathcal{T}}_{c(1,b)} S^{\downarrow}_d(0)
+ {\mathcal{T}}_{c(2,b)} S^{\downarrow}_r(0)
\end{eqnarray}

Defined by ,

\begin{eqnarray}
  {\mathcal{T}}_{c(2,b)}= \{ ( 1 - A_2 )( A_1 - \alpha_{0(2,b)} )
                      - ( 1 - A_1 )( A_2 - \alpha_{0(2,b)} ) \} / A_3 \\
 {\mathcal{T}}_{c(1,b)}= - C_3 {\mathcal{T}}_{c(2,b)}  \\
 +               \{ ( C_3 \alpha_{0(2,b)} + \alpha_{0(1,b)} -C_4 )
                   ( ( 1 - A_1 ) e^{ a LAI }
                   - ( 1 - A_2 ) e^{-a LAI } )  / A_3
                   + C_3 - C_4 +1 \} e^{- F\cdot LAI\cdot \sec\zeta}
 \\
\end{eqnarray}

The above is performed for \(b=1, 2\) (visible and near-infrared),
respectively. The above procedure is performed for \(b=1, 2\) (visible
and near-infrared), respectively.

The reflectance \(r_f\) and transmission \(t_f\) are read as external
parameters for each land cover type, but before they are used in the
above calculations, the following two modifications are made.

\begin{enumerate}
\def\labelenumi{\arabic{enumi}.}
\tightlist
\item
  the effect of snow (ice) on the leaves When the canopy temperature is
  less than 0 \(^{\circ}\) C, the water above the canopy is regarded as
  snow (ice). In this case, the snow albedo (\(\alpha_{Sn(b)}\)) and the
  water content in the canopy (\(w_c\)) are used to determine the snow
  (ice),
\end{enumerate}

\begin{eqnarray}
 r_{f(b)} = ( 1 - f_{cwet} ) r_{f(b)}
         + f_{cwet} \alpha_{Sn(b)} \\
  f_{cwet} = {w_c}/w_{c,cap}
\end{eqnarray}

The following table shows the volume of water on the canopy.
\(w_{c,cap}\) is the water content in the canopy. The transmittance is
given in the following formula for convenience to avoid negative
absorption (\(1-r_f-t_f\)), i.e.

\begin{eqnarray}
 t_{f(b)} = ( 1 - f_{cwet} ) t_{f(b)}
         + f_{cwet} t_{Sn(b)}, \ \ \
 t_{Sn(b)} = \min( 0.5(1 - \alpha_{Sn(b)}), t_{f(b)} )
\end{eqnarray}

When the water on the canopy is liquid, we should ignore the change in
the radiative parameters of the leaf surface. When the liquid water on
the canopy is liquid, changes in the radiative parameters of the leaf
surface due to the liquid water on the canopy are ignored. In the case
of snowfall trapping by the canopy (snow), and in the case of freezing
of the liquid water on the canopy (ice), the same albedo as that of the
snow cover on the forest floor is used here, although the radiative
characteristics of each case may be different.

\begin{enumerate}
\def\labelenumi{\arabic{enumi}.}
\setcounter{enumi}{1}
\tightlist
\item
  the effect of considering the direction of reflection and transmission
  In the solution of the above equation, it is assumed that all the
  reflected light returns to the direction of the incident light, but
  considering that some of it is scattered in the same direction as the
  incident light, we can replace the radial parameters of the leaf
  surface with the following (Watanabe, in press).
\end{enumerate}

\begin{eqnarray}
  r_{f(b)} = 0.75 r_{f(b)} + 0.25 t_{f(b)} \\
  t_{f(b)} = 0.75 t_{f(b)} + 0.25 r_{f(b)}
\end{eqnarray}

The above is done for \(b=1, 2\) (visible and near-infrared),
respectively.

We also take into account cases where vegetation is unevenly distributed
in parts of the grid (e.g., savannahs), prior to the calculation of
albedo, etc,

\begin{eqnarray}
  LAI = LAI / f_V
\end{eqnarray}

The LAI of the vegetation cover (the original LAI is considered to be
the grid average) is calculated as the LAI of the vegetation cover,
which is used in the calculation of the albedo described above. \(f_V\)
is the vegetation coverage of the grid. After the albedo and other data
are obtained, the LAI of the grid is calculated by

\begin{eqnarray}
  \alpha_{s(d,b)} = f_V \alpha_{s(d,b)}
                       + ( 1 - f_V ) \alpha_{0(d,b)} \\
  {\mathcal{T}}_{c(d,b)} = f_V {\mathcal{T}}_{c(d,b)}
                       + ( 1 - f_V ) ( 1 - \alpha_{0(d,b)} )
\end{eqnarray}

We take the area-weighted average of the vegetation-covered and
non-vegetation-covered portions, as

\hypertarget{calculations-such-as-surface-radiation-flux}{%
\subsubsection{Calculations such as surface radiation
flux}\label{calculations-such-as-surface-radiation-flux}}

Using the downward radiation flux (\(R^{\downarrow}_{(d,b)}\)) and the
albedo obtained above, the following radiation fluxes are obtained.

\begin{eqnarray}
 R^{\downarrow}_S = \sum_{b=1}^2\sum_{d=1}^2 R^{\downarrow}_{(d,b)} \\
 R^{\uparrow}_S = \sum_{b=1}^2\sum_{d=1}^2 \alpha_{s(d,b)} R^{\downarrow}_{(d,b)} \\
 R^{\downarrow}_L = R^{\downarrow}_{(2,3)} \\
 R^{gnd}_S = \sum_{b=1}^2\sum_{d=1}^2 {\mathcal{T}}_{s(d,b)} R^{\downarrow}_{(d,b)} \\
 PAR = \sum_{d=1}^2 R^{\downarrow}_{(d,1)}
\end{eqnarray}

\(R^{\downarrow}_S\) and \(R^{\uparrow}_S\) represent the downward and
upward shortwave radiation fluxes, \(R^{\downarrow}_L\) represents the
downward longwave radiation flux, \(R^{gnd}_S\) represents the shortwave
radiation flux absorbed by the forest floor, and \(PAR\) represents the
downward Photosynthetically Active Radiation (PAR) flux .

The canopy transmittance of the shortwave and longwave canopies and the
longwave emission coefficient are obtained as follows.

\begin{eqnarray}
 {\mathcal{T}}_{cS} = R^{gnd}_S / ( R^{\downarrow}_S - R^{\uparrow}_S ) \\
 {\mathcal{T}}_{cL} = \exp( - F \cdot LAI \cdot d_f ) \\
 \epsilon = 1 - \alpha_{s(2,3)}
\end{eqnarray}

\hypertarget{calculation-of-turbulent-parameters-bulk-coefficients.}{%
\subsection{Calculation of turbulent parameters (bulk
coefficients).}\label{calculation-of-turbulent-parameters-bulk-coefficients.}}

Calculate the turbulence parameters (bulk coefficient).

\hypertarget{roughness-calculations-for-momentum-and-heat.}{%
\subsubsection{roughness calculations for momentum and
heat.}\label{roughness-calculations-for-momentum-and-heat.}}

The calculation of roughness is based on Watanabe (1994). Using the
results of Kondo and Watanabe's (1992) multi-layered canopy model,
Watanabe (1994) proposed the following roughness functions for the bulk
model that best fit the results.

\begin{eqnarray}
 \left(\ln \frac{h-d}{z_0}\right)^{-1} =
 \left[ 1 - \exp( -A^+) + \left(-\ln \frac{z_{0s}}{h}\right)^{-1/0.45}
  \exp(-2A^+)\right]^{0.45} \\
 \left(\ln \frac{h-d}{z_T^{\dagger}}\right)^{-1} =
 \frac{1}{-\ln(z_{Ts}/h)} \left[ \frac{P_1}{P_1 + A^+ \exp({A^+})}\right] ^{P2} \\
 \left(\ln \frac{h-d}{z_0}\right)^{-1} \left(\ln \frac{h-d}{z_T}\right)^{-1}
 = C_T^{\infty} \left[1-\exp(-P_3 A^+)
  + \left(\frac{C_T^0}{C_T^{\infty}}\right)^{1/0.9} \exp(-P_4 A^+)\right]^{0.9} \\
 h-d = h [1-\exp(-A^+)] / {A^+} \\
 A^+ = \frac{c_d LAI}{2k^2} \\
 \frac1{C_T^0} = \ln \frac{h-d}{z_0} \ln \frac{h-d}{z_T^{\dagger}} \\
 C_T^{\infty} = \frac{-1+(1+8F_T)^{1/2}}{2} \\
 P_1 = 0.0115 \left(\frac{z_{Ts}}{h}\right)^{0.1}
  \exp\left[5 \left(\frac{z_{Ts}}{h}\right)^{0.22}\right] \\
 P_2 = 0.55 \exp\left[-0.58 \left(\frac{z_{Ts}}{h}\right)^{0.35}\right] \\
 P_3 = [F_T + 0.084 \exp(-15 F_T)]^{0.15} \\
 P_4 = 2 F_T^{1.1} \\
 F_T = c_h / c_d
\end{eqnarray}

where \(z_0\) and \(z_T\) are the roughness of the entire canopy with
respect to momentum and heat, respectively, \(z_0s\) and \(z_Ts\) are
the roughness of the ground surface (forest floor) with respect to
momentum and heat, respectively, \(c_d\) and \(c_h\) are the roughness
of the ground surface with respect to momentum and heat, respectively
\(h\) is the vegetation height, \(d\) is the zero-plane displacement,
and \(LAI\) is the LAI for the heat transfer coefficient between the
individual leaves and the atmosphere for the \(z_T^{\dagger}\) is the
roughness of the heat flux at the leaf surface under the assumption of
no heat flux, and is used to determine the heat transport coefficient
from the forest floor.

\(z_{0s}\) and \(z_{Ts}\) are given as external data for each land cover
type, but the standard values (\(z_{0s}=0.05\) m, \(z_{Ts}=0.005\) m)
are fixed regardless of land cover type. However, for snow cover, the
following modifications are made.

\begin{eqnarray}
 z_{0s} = \max( f_{Sn} z_{0s}, z_{0Sn} ) \\
 z_{Ts} = \max( f_{Sn} z_{0s}, z_{TSn} ) \\
          f_{Sn} = 1 - D_{Sn} / z_{0s}
\end{eqnarray}

where \(D_{Sn}\), \(z_{0Sn}\) and \(z_{TSn}\) are the roughness of the
snow surface with respect to momentum and heat, respectively.

\(c_d\) and \(c_h\) are parameters determined by the shape of the leaves
and are given as external data for each land cover type.

\hypertarget{calculating-bulk-coefficients-for-momentum-and-heat}{%
\subsubsection{Calculating Bulk Coefficients for Momentum and
Heat}\label{calculating-bulk-coefficients-for-momentum-and-heat}}

The bulk coefficients are also derived following Watanabe (1994), using
Monin-Obukhov's law of similarity, as follows

\begin{eqnarray}
 C_M = k^2 \left[ \ln \frac{z_a-d}{z_0} + \Psi_m(\zeta) \right]^{-2} \\
 C_H = k^2 \left[ \ln \frac{z_a-d}{z_0} + \Psi_m(\zeta) \right]^{-1}
             \left[ \ln \frac{z_a-d}{z_T} + \Psi_h(\zeta) \right]^{-1} \\
 C_{Hs} = k^2 \left[ \ln \frac{z_a-d}{z_0} + \Psi_m(\zeta_g) \right]^{-1}
             \left[ \ln \frac{z_a-d}{z_T^{\dagger}} + \Psi_h(\zeta_g) \right]^{-1} \\
 C_{Hc} = C_H - C_{Hs}
\end{eqnarray}

where \(C_M\) and \(C_H\) are the bulk coefficient of total canopy
(foliage \(+\) forest floor) for momentum and heat, respectively,
\(C_{Hs}\) is the bulk coefficient of surface (forest floor) flux for
heat, and \(C_{Hc}\) is the bulk coefficient of canopy (leaf surface)
flux for heat. Bulk coefficients, \(\Psi_m\) and \(\Psi_h\) are
Monin-Obukhov shear functions for momentum and heat, respectively, and
\(z_a\) is the reference altitude of the atmosphere (altitude of the
first layer of the atmosphere). \(\zeta\) and \(\zeta_g\) are calculated
using the Monin-Obukhov length \(L\) and \(L_g\) for the whole canopy
and ground surface (forest floor), respectively,

\begin{eqnarray}
 \zeta = \frac{z_a - d}{L} \\
 \zeta_g = \frac{z_a - d}{L_g}
\end{eqnarray}

where the length is denoted by The Monin-Obukhov length is expressed in

\begin{eqnarray}
 L = \frac{\Theta_0 C_M^{3/2}|V_a|^2}{kg(C_{Hs}(T_s - T_a) + C_{Hc}(T_c - T_a))} \\
 L_s = \frac{\Theta_0 C_M^{3/2}|V_a|^2}{kg C_{Hs}(T_s - T_a)}
\end{eqnarray}

is expressed in where \(\Theta_0\) =300K, \(|V_a|\) is the absolute
surface wind speed, \(k\) is the Kalman constant, \(g\) is the
acceleration due to gravity, \(T_a\), \(T_c\), and \(T_s\) are the first
atmospheric layer, the canopy (leaf surface) and the ground (forest
floor), respectively ) temperature.

Since the bulk coefficients are required for the calculation of the
Monin-Obukhov length and the Monin-Obukhov length is required for the
calculation of the bulk coefficients, the neutral bulk coefficients are
used as the initial values and are repeated (twice in the standard
case).

Prior to this calculation, the snow depth is added to the zero-plane
displacements, but the zero-plane displacements should be limited to a
maximum value so that they are not too large compared to the \(z_a\).

\begin{eqnarray}
 d = \min( d + D_{Sn} ,\  f_{\max} \cdot z_a )
\end{eqnarray}

The standard is taking \(f_{\max}\) to 0.5.

\hypertarget{calculating-bulk-coefficients-for-water-vapor}{%
\subsubsection{Calculating Bulk Coefficients for Water
Vapor}\label{calculating-bulk-coefficients-for-water-vapor}}

This calculation is performed after the calculation of the stomatal
resistance, which is described below.

Once the stomatal resistance (\(r_{st}\)) and surface evaporation
resistance (\(r_{soil}\)) are obtained, the bulk coefficient for water
vapor is calculated as follows

\begin{eqnarray}
 C_{Ec} |V_a| = \left[ (C_{Hc} |V_a|)^{-1} + r_{st} / LAI\right]^{-1} \\
 C_{Es} |V_a| = \left[ (C_{Hs} |V_a|)^{-1} + r_{soil}\right]^{-1}
\end{eqnarray}

(Previously, the pore resistance was calculated via roughness by
converting the pore resistance into a reduction in the exchange
coefficient, but this method was changed to a simpler method because it
seemed to be problematic.)

Note that the bulk coefficient of water vapor is the same as the bulk
coefficient of heat when no stomatal resistance is applied (e.g.,
evaporation from a wet surface).

\hypertarget{calculation-of-stomatal-resistance.}{%
\subsection{calculation of stomatal
resistance.}\label{calculation-of-stomatal-resistance.}}

The calculation of the stomatal resistance is based on the
photosynthesis-stomatal model based on Farquhar et al.~(1980), Ball
(1988), Collatz et al.~The code of SiB2 (Sellers et al., 1996) is used
almost verbatim, except for the method to determine the resistance of
the entire canopy. It is also possible to use Jarvis-type empirical
equations instead, but the explanation is omitted here.

\hypertarget{calculating-soil-moisture-stress-factors.}{%
\subsubsection{Calculating Soil Moisture Stress
Factors.}\label{calculating-soil-moisture-stress-factors.}}

Determine the soil water stress on transpiration. The soil water stress
factor for each soil layer is determined and the overall soil stress
factor is calculated by weighting the stress factor by the distribution
of roots in each layer.

Soil water stress in each layer is assessed using SiB2 (Sellers et al.,
1996) as a guide, using the following equation

\begin{eqnarray}
 f_{w(k)} = [ 1 + \exp( 0.02 (\psi_{cr} - \psi_{k}) ) ]^{-1}
\ \ \ \ \ (k=1,\ldots,K_g)
\end{eqnarray}

Overall soil stressors are ,

\begin{eqnarray}
 f_w = \sum_{k=1}^{K_g} f_{w(k)} f_{root(k)}
\end{eqnarray}

Here, \(f_{root(k)}\) is the ratio of the presence of roots in each
layer and is an external parameter for each land cover type. This is
\(\sum_{k=1}^{K_g} f_{root(k)}=1\).

In addition, the weights that distribute the transpiration to the
siphoning flux in each layer are calculated as follows.

\begin{eqnarray}
 f_{rootup(k)} = f_{w(k)} f_{root(k)} / f_w
\ \ \ \ \ (k=1,\ldots,K_g)
\end{eqnarray}

Note that it is \(\sum_{k=1}^{K_g} f_{rootup(k)} = 1\).

\hypertarget{calculating-photosynthesis}{%
\subsubsection{Calculating
Photosynthesis}\label{calculating-photosynthesis}}

Following SiB2 (Sellers et al., 1996), we calculate the amount of
photosynthesis.

We believe that the amount of photosynthesis is determined by three
upper limits.

\begin{eqnarray}
 A \leq \min( w_c, w_e, w_s) 
\end{eqnarray}

\(w_c\) is the upper limit of photosynthetic enzyme (Rubisco)
efficiency, and \(w_e\) is the upper limit of photosynthetically
available radiation. \(w_s\) is the upper limit of the sink of
photosynthetic products in the case of plants, and C\(_4\) is the upper
limit of the concentration of CO\(_2\) in the case of plants (Collatz et
al., 1991, 1992).

The size of each is estimated as follows.

\begin{eqnarray}
 w_c = \left\{
\begin{array}{ll}
\displaystyle{
V_m \left[ \frac{c_i - \Gamma^*}{c_i + K_c(1+O_2/K_O)}\right]
}
  ({C$_3$ 植物の場合})\\
 V_m
  ({C$_4$ 植物の場合})
\end{array}
\right. \\
 w_e = \left\{
\begin{array}{ll}
\displaystyle{
PAR\cdot \epsilon_3 \left[ \frac{c_i-\Gamma^*}{c_i+2\Gamma^*}\right]
}
  ({C$_3$ 植物の場合})\\
PAR\cdot \epsilon_4
  ({C$_4$ 植物の場合})
\end{array}
\right. \\
 w_s = \left\{
\begin{array}{ll}
V_m / 2
  ({C$_3$ 植物の場合})\\
V_m c_i/ 5
  ({C$_4$ 植物の場合})
\end{array}
\right.
\end{eqnarray}

where \(V_m\) is the Rubisco reaction capacity, \(c_i\) is the partial
pressure of CO\(_2\) in the stomatal chamber, \(O_2\) is the partial
pressure of oxygen in the stomatal chamber, and \(PAR\) is the
photosynthetically available radiation (PAR). \(\Gamma^*\) is the
compensation point of CO\(_2\) and is represented by
\(\Gamma^* = 0.5 O_2 / S\) \(K_c\), \(K_O\), and \(S\) are functions of
temperature and will be shown in functional form later. \(\epsilon_3\)
and \(\epsilon_4\) are constants determined by vegetation type.

\begin{enumerate}
\def\labelenumi{(\arabic{enumi})}
\setcounter{enumi}{75}
\tightlist
\item
  is actually solved as follows to represent a smooth transition between
  the different upper bounds.
\end{enumerate}

\begin{eqnarray}
 \beta_{ce} w_p^2 - w_p(w_c + w_e) + w_c w_e = 0 \\
 \beta_{ps} A^2 - A(w_p + w_s) + w_p w_s = 0
\end{eqnarray}

Solving the two equations in sequence, choosing the smaller of the two
solutions for each equation, yields \(A_n\). \(\beta_{ce}, \beta_{ps}\)
are constants determined by vegetation type. Note that when \(\beta=1\),
it coincides with a simple minimum value operation.

Once the amount of photosynthesis is determined, the net photosynthesis
amount (\(A_n\)) is determined as follows.

\begin{eqnarray}
 A_n = A - R_d
\end{eqnarray}

\(R_d\) is the respiratory rate and is expressed as

\begin{eqnarray}
 R_d = f_d V_m
\end{eqnarray}

\(f_d\) is a constant determined by vegetation type.

\(V_m\), for example, depends on temperature and soil moisture as
follows (\(V_m\) depends differently on temperature depending on the
term that appears, but is represented by the same \(V_m\)).

\begin{eqnarray}
 V_m = V_{\max} f_T(T_c) f_w \\
 K_c = 30 \times 2.1^{Q_T} \\
 K_O = 30000 \times 1.2^{Q_T} \\
 S   = 2600 \times 0.57^{Q_T} \\
 f_T(T_c) = \left\{
\begin{array}{ll}
 2.1^{Q_T}/\{1 + \exp[s_1(T_c-s_2)]\}  ({C$_3$ の $w_c$, $w_e$ のとき})\\
 1.8^{Q_T}/\{1 + \exp[s_3(s_4-T_c)]\}  ({C$_3$ の $w_s$ のとき}) \\
 2.1^{Q_T}/\{1 + \exp[s_1(T_c-s_2)]\}/\{1 + \exp[s_3(s_4-T_c)]\}
    ({C$_4$ の $w_c$, $w_e$ のとき})\\
 1.8^{Q_T}   ({C$_4$ の $w_s$ のとき}) \\
 2^{Q_T}/\{1 + \exp[s_5(T_c-s_6)]\}   ({$R_d$ のとき})
\end{array}
\right. \\
Q^T = (T_c - 298) / 10
\end{eqnarray}

Here, \(V_{\max}\), \(s_1, \ldots, s_6\) are constants determined by the
vegetation type.

Given the above values for \(V_{\max}\), \(PAR\), \(c_i\), \(T_c\) and
\(f_w\), the amount of photosynthesis in each individual leaf can be
calculated. In reality, these values can be considered to be distributed
evenly within the same canopy, but we assumed that \(c_i\), \(T_c\), and
\(f_w\) are the same for all leaves and that the vertical distributions
of \(V_{\max}\) and \(PAR\) are considered. The vertical distribution of
the \(PAR\) is large at the top of the canopy and diminishes as it moves
downward, and the distribution of the \(V_{\max}\) is considered to be
similar to that of the \(PAR\) following this distribution of the
\(PAR\).

The average vertical distribution of the \(PAR\) (and therefore the
vertical distribution of the \(V_{\max}\)) is shown in the following
table.

\begin{eqnarray}
 PAR(L) = PAR^{top} \exp(- f_{atn} a L)
\end{eqnarray}

Here, \(L\) is the leaf area accumulated from the top of the canopy,
\(PAR^{top}\) is the \(PAR\) at the top of the canopy, \(a\) is the
attenuation factor defined in (17), and \(f_{atn}\) is a constant for
adjustment. Using this value, the factor (\(f_{avr}\)) which represents
the average value of \(PAR\) is defined as follows.

\begin{eqnarray}
 f_{avr} = \int_0^{LAI} PAR(L) dL \Bigm / (LAI \cdot PAR^{top})
 = \frac{1 - \exp(- f_{atn} a L)}{f_{atn} a}
\end{eqnarray}

Since each term in \(A_n\) (\(w_c, w_s, w_e, R_d\)) is proportional to
\(V_{\max}\) or \(PAR\), on the assumption that the vertical
distributions of \(V_{\max}\) and \(PAR\) are proportional to those of
\(V_{\max}\) at the top end of the canopy By multiplying the \(A_n\)
calculated using the \(PAR\) value by \(f_{avr}\), the average leaf
photosynthetic rate (\(\overline{A_n}\)) can be obtained.

\begin{eqnarray}
 \overline{A_n} = f_{avr} A_n
\end{eqnarray}

Hereinafter, we will refer to it again as \(A_n\).

\hypertarget{stomatal-resistance-calculations.}{%
\subsubsection{Stomatal Resistance
Calculations.}\label{stomatal-resistance-calculations.}}

Net photosynthesis (\(A_n\)) and stomatal conductance (\(g_s\)) are
related by the quasi-empirical formula of Ball (1988) as follows

\begin{eqnarray}
 g_s = m \frac {A_n}{c_s} h_s + b f_w
\end{eqnarray}

where \(c_s\) is the CO\(_2\) molar fraction at the leaf surface (the
number of mol of CO\(_2\) per 1mol of air), \(f_w\) is the soil moisture
stress factor, and \(m\) and \(b\) are constants determined by
vegetation type.

\(h_s\) is the relative humidity at the leaf surface, defined as

\begin{eqnarray}
 h_s = e_s / e_i
\end{eqnarray}

\(e_s\) is the mole fraction of water vapor at the leaf surface, \(e_i\)
is the mole fraction of water vapor in the stomata, and
\(e_i = e^*(T_c)\) is the mole fraction of water vapor in the stomata.
\(e^*\) is the mole fraction of saturated water vapor.

Assuming that the water vapor flux from the stomatal surface to the leaf
surface is equal to the water vapor flux from the leaf surface to the
atmosphere (i.e., there is no convergent water vapor divergence at the
leaf surface),

\begin{eqnarray}
 g_s(e_i - e_s) = g_l(e_s - e_a)
\end{eqnarray}

than ,

\begin{eqnarray}
 e_s = ( g_l e_a + g_s e_i ) / ( g_l + g_s )
\end{eqnarray}

is obtained. where \(e_a\) is the atmospheric water vapor mole fraction
and \(g_l\) is the conductance between the leaf surface and the
atmosphere. \(g_l\) is represented by \(g_l = C_{Hc}|V_a| / LAI\) using
the bulk coefficient.

Similarly, given the lack of convergent divergence of CO\(_2\) on the
leaf surface,

\begin{eqnarray}
 A_n = g_l(c_a - c_s)/1.4
     = g_s(c_s - c_i)/1.6
\end{eqnarray}

than ,

\begin{eqnarray}
 c_s = c_a - 1.4 A_n/g_l \\
 c_i = c_s - 1.6 A_n/g_s
\end{eqnarray}

where \(c_a\) and \(c_i\) are obtained from the atmosphere and pores,
respectively. where \(c_a\) and \(c_i\) are the CO\(_2\) mole fractions
in the atmosphere and in the pores, respectively. 1.4 and 1.6 are
constants that appear due to the difference in diffusion coefficients of
water vapor and CO\(_2\).

Substituting (94) and (96) into (93), we obtain the following equation
for \(g_s\).

\begin{eqnarray}
 H g_s^2 + ( H g_l - e_i - H b f_w ) g_s - g_l ( H b f_w + e_a ) = 0
\end{eqnarray}

However,

\begin{eqnarray}
 H = (e_i c_s)/(m A_n)
\end{eqnarray}

and use (99) for \(c_s\).

Of the two solutions of (100), the larger of the two solutions makes
more sense. From the above, assuming that \(A_n\) is known, we can solve
for \(g_s\), but we use \(c_i\) to solve for \(A_n\). \(c_i\) can be
obtained by (99) if \(g_s\) is obtained. In other words, finding \(g_s\)
requires \(A_n\) and finding \(A_n\) requires \(c_i\) and thus \(g_s\),
so the calculation must be repeated.

The algorithm for iterative computation is ported from SiB2. Six
iterations are performed and the next solution is estimated in the order
of increasing errors to accelerate the convergence.

Finally, using the stomatal conductance, the stomatal resistance is
expressed as

\begin{eqnarray}
 r_{st} = 1/g_{st}
\end{eqnarray}

\hypertarget{calculation-of-surface-evaporation-resistance}{%
\subsubsection{Calculation of Surface Evaporation
Resistance}\label{calculation-of-surface-evaporation-resistance}}

Calculate the surface evaporation resistance (\(r_{soil}\)) and the
relative humidity (\(h_{soil}\)) of the first layer of soil as follows.

\begin{eqnarray}
 r_{soil} = a_1 ( 1 - W_{(1)} ) / ( a_2 + W_{(1)} ) \\
 h_{soil} = \exp \left(\frac{\psi_{(1)} g}{R_{air} T_{g(1)}} \right)
\end{eqnarray}

where \(W_{(1)} = w_{(1)}/w_{sat(1)}\) is the saturation degree of the
first soil layer, \(\psi_{1}\) is the moisture potential of the first
soil layer, \(g\) is the gravitational acceleration, \(R_{air}\) is the
gas constant of air, and \(T_{g(1)}\) is the temperature of the first
soil layer. \(a_1\) and \(a_2\) are constants and the standard uses
\(a_1=800\), \(a_2\)=0.2.

\hypertarget{earth-surface-submodel-matsfc}{%
\section{Earth Surface Submodel
MATSFC}\label{earth-surface-submodel-matsfc}}

\hypertarget{calculation-of-surface-turbulence-flux.}{%
\subsection{Calculation of surface turbulence
flux.}\label{calculation-of-surface-turbulence-flux.}}

The bulk method is used to obtain the turbulent flux at the surface as
follows. When the surface energy balance is solved and the surface
temperature (\(T_s\)) and the canopy temperature (\(T_c\)) are updated,
the surface flux is also updated with respect to these values. The value
obtained here is a provisional value until then. In order to linearize
the energy balance for \(T_s\) and \(T_c\), the derivatives of each flux
for \(T_s\) and \(T_c\) have been calculated.

\begin{itemize}
\tightlist
\item
  momentum flux
\end{itemize}

\begin{eqnarray}
 \tau_x = - \rho C_{M}|V_a| u_a \\
 \tau_y = - \rho C_{M}|V_a| v_a
\end{eqnarray}

Here, \(\tau_x\) and \(\tau_y\) are the momentum fluxes (surface
stresses) in the east-west and north-south directions, respectively.

\begin{itemize}
\tightlist
\item
  Sensible Heat Flux
\end{itemize}

\begin{eqnarray}
 H_s = c_p \rho C_{Hs}|V_a| (T_s - (P_s/P_a)^{\kappa}T_a)
  \\
 H_c = c_p \rho C_{Hc}|V_a| (T_c - (P_s/P_a)^{\kappa}T_a) \\
 \partial H_s/\partial T_s = c_p \rho C_{Hs}|V_a| \\
 \partial H_c/\partial T_c = c_p \rho C_{Hc}|V_a|
\end{eqnarray}

where \(H_s\) and \(H_c\) are sensible heat fluxes from the ground
surface (forest floor) and canopy (leaf surface), respectively,
\(\kappa = R_{air} / c_p\) and \(R_{air}\) are the gas constant of air,
and \(c_p\) is the specific heat of air.

\begin{itemize}
\tightlist
\item
  Bare ground (forest floor) evaporation flux
\end{itemize}

\begin{eqnarray}
 Et_{(1,1)} = (1-A_{Sn})(1-f_{ice})\cdot
           \rho \widetilde{C_{Es}}|V_a|(h_{soil}q^*(T_s) - q_a) \\
 Et_{(2,1)} = (1-A_{Sn})f_{ice}\cdot
           \rho \widetilde{C_{Es}}|V_a|(h_{soil}q^*(T_s) - q_a) \\
 \partial Et_{(1,1)}/\partial T_s = (1-A_{Sn})(1-f_{ice})\cdot
           \rho \widetilde{C_{Es}}|V_a|h_{soil}\cdot dq^*/dT |_{T_s} \\
 \partial Et_{(2,1)}/\partial T_s = (1-A_{Sn})f_{ice}\cdot
           \rho \widetilde{C_{Es}}|V_a|h_{soil}\cdot dq^*/dT |_{T_s}
\end{eqnarray}

where \(Et_{(1,1)}\) and \(Et_{(2,1)}\) are water evaporation and ice
sublimation fluxes on bare ground, respectively, \(q^*(T_s)\) is the
specific humidity at the saturated surface temperature, \(h_{soil}\) is
the relative humidity at the soil surface, \(A_{Sn}\) is the snow cover
area fraction, and \(f_{ice}\) is the percentage of ice in the first
soil layer

\begin{eqnarray}
  f_{ice} = w_{i(1)}/w_{(1)}
\end{eqnarray}

in \(A_{Sn}\). Note that the value of \(A_{Sn}\) should be either \(0\)
(snow-free surface) or \(1\) (snow-covered surface) because snow-free
and snow-covered surfaces are calculated separately. In the case of
downward-facing (condensation) fluxes, no soil moisture resistance is
applied, so the bulk coefficient should be calculated as follows

\begin{eqnarray}
  \widetilde{C_{Es}} = \left\{
  \begin{array}{ll}
   C_{Es} (h_{soil}q^*(T_s) - q_a > 0 {のとき})\\
   C_{Hs} (h_{soil}q^*(T_s) - q_a \leq 0 {のとき})
  \end{array}
  \right.
\end{eqnarray}

\begin{itemize}
\tightlist
\item
  Transpiration Flux
\end{itemize}

\begin{eqnarray}
 Et_{(1,2)} = (1-f_{cwet}) \cdot \rho \widetilde{C_{Ec}}|V_a|(q^*(T_c) - q_a) \\
 Et_{(2,2)} = 0 \\
 \partial Et_{(1,2)}/\partial T_c =
  (1-f_{cwet}) \cdot \rho \widetilde{C_{Ec}}|V_a|\cdot dq^*/dT|_{T_c} \\
 \partial Et_{(2,2)}/\partial T_c = 0
\end{eqnarray}

Here, \(Et_{(1,2)}\) and \(Et_{(2,2)}\) are water and ice transpiration,
while \(Et_{(2,2)}\) is always zero. \(f_{cwet} = w_c / w_{c,cap}\) is
the wetting area ratio of the canopy. In the case of downward-facing
fluxes, the bulk factor is considered to be condensation on dry parts of
the leaves and is calculated as follows

\begin{eqnarray}
  \widetilde{C_{Ec}} = \left\{
  \begin{array}{ll}
   C_{Ec} (q^*(T_c) - q_a > 0 {のとき})\\
   C_{Hc} (q^*(T_c) - q_a \leq 0 {のとき})
  \end{array}
  \right.
\end{eqnarray}

\begin{itemize}
\tightlist
\item
  Evaporated flux on the canopy \(T_c\) \(\geq\) 0 \(^{\circ}\) C
\end{itemize}

\begin{eqnarray}
 Et_{(1,3)} =
  f_{cwet} \cdot \rho C_{Hc}|V_a|(q^*(T_c) - q_a) \\
 Et_{(2,3)} = 0 \\
 \partial Et_{(1,3)} \partial T_c =
  f_{cwet} \cdot \rho C_{Hc}|V_a|\cdot dq^*/dT|_{T_c} \\
 \partial Et_{(2,3)} \partial T_c = 0
\end{eqnarray}

\begin{verbatim}
 $T_c$ $<$ 0 $^{\circ}$ In case of C
\end{verbatim}

\begin{eqnarray}
 Et_{(1,3)} = 0 \\
 Et_{(2,3)} =
  f_{cwet} \cdot \rho C_{Hc}|V_a|(q^*(T_c) - q_a) \\
 \partial Et_{(1,3)} \partial T_c = 0 \\
 \partial Et_{(2,3)} \partial T_c =
  f_{cwet} \cdot \rho C_{Hc}|V_a|\cdot dq^*/dT|_{T_c}
\end{eqnarray}

Here, \(Et_{(1,3)}\) and \(Et_{(2,3)}\) are the evaporation of water and
ice sublimation on the canopy.

\begin{itemize}
\tightlist
\item
  Snow Sublimation Flux
\end{itemize}

\begin{eqnarray}
 E_{Sn} = A_{Sn}\cdot \rho C_{Hs}|V_a|(q^*(T_s) - q_a) \\
 \partial E_{Sn}/\partial T_s = A_{Sn}\cdot \rho C_{Hs}|V_a|
 \cdot dq^*/dT|_{T_s}
\end{eqnarray}

\begin{verbatim}
 $E_{Sn}$ is a snow sublimation flux. Note that since snow-free and snow-covered surfaces are calculated separately, $A_{Sn}$ contains either $0$ (snow-free surface) or $1$ (snow-covered surface).
\end{verbatim}

\hypertarget{calculating-heat-transfer-flux.}{%
\subsection{Calculating heat transfer
flux.}\label{calculating-heat-transfer-flux.}}

Calculating the heat conduction fluxes on snow-free and snow-covered
surfaces. As well as the turbulent fluxes, the heat conduction fluxes
are also updated when the surface temperature is updated after the
energy balance is solved.

Note that since snow-free and snow-covered surfaces are calculated
separately, the snow coverage area ratio (\(A_{Sn}\)) should be set to
either \(0\) (snow-free surface) or \(1\) (snow-covered surface), as
shown below.

\begin{itemize}
\tightlist
\item
  Heat Transfer Flux on Snow-Free Surfaces
\end{itemize}

\begin{eqnarray}
  F_{g(1/2)} = (1 - A_{Sn}) \cdot k_{g(1/2)} / \Delta z_{g(1/2)} (T_{g(1)} - T_s) \\
  \partial F_{g(1/2)}/\partial T_s =
  - (1 - A_{Sn}) \cdot k_{g(1/2)} / \Delta z_{g(1/2)}
\end{eqnarray}

where \(F_{g(1/2)}\) is the heat transfer flux, \(k_{g(1/2)}\) is the
thermal conductivity of the soil, \(\Delta z_{g(1/2)}\) is the thickness
of the first layer of soil from the temperature definition point to the
ground surface, and \(T_{g(1)}\) is the temperature of the first layer
of soil.

\begin{itemize}
\tightlist
\item
  Heat Transfer Flux of Snow Surface
\end{itemize}

\begin{eqnarray}
  F_{Sn(1/2)} = A_{Sn} \cdot k_{Sn(1/2)} / \Delta z_{Sn(1/2)} (T_{Sn(1)} - T_s)
 \\
  \partial F_{Sn(1/2)}/\partial T_s =
  - A_{Sn} \cdot k_{Sn(1/2)} / \Delta z_{Sn(1/2)}
\end{eqnarray}

where \(F_{Sn(1/2)}\) is the heat transfer flux, \(k_{Sn(1/2)}\) is the
thermal conductivity of the snowpack, \(\Delta z_{Sn(1/2)}\) is the
thickness of the first layer of snow from the temperature definition
point to the ground surface, and \(T_{Sn(1)}\) is the temperature of the
first layer of snowpack.

\hypertarget{surface-solving-the-canopy-energy-balance}{%
\subsection{Surface, Solving the Canopy Energy
Balance}\label{surface-solving-the-canopy-energy-balance}}

The energy balance is solved for two cases: 1: the case of no melting
and 2: the case of melting at the surface. In Case 2, we fix the surface
temperature (\(T_s\)) to \(0^{\circ}\) C, and diagnose the energy
available for melting based on the energy balance. Since the snow
melting on vegetation will be processed by correction later, we do not
solve this case separately here. The case in which the snow melts
completely within a time step is also processed by later corrections.

\hypertarget{surface-canopy-energy-balance}{%
\subsubsection{Surface, Canopy Energy
Balance}\label{surface-canopy-energy-balance}}

The amount of energy dissipation at the ground surface (forest floor) is
,

\begin{eqnarray}
 \Delta F_s =
  H_s + R^{net}_s + l Et_{(1,1)} + l_s ( Et_{(2,1)} + E_{Sn} )
  - F_{g(1/2)} - F_{Sn(1/2)}
\end{eqnarray}

However, \(l\) and \(l_s\) are the latent heat of evaporation and
sublimation, respectively, and \(R^{net}_s\) is the net radiative
divergence at the ground surface,

\begin{eqnarray}
  R^{net}_s = -(R^{\downarrow}_S - R^{\uparrow}_S) {\mathcal{T}}_{cS}
              - \epsilon R^{\downarrow}_L {\mathcal{T}}_{cL}
              + \epsilon \sigma T_s^4
              - \epsilon \sigma T_c^4 (1 - {\mathcal{T}}_{cL})
\end{eqnarray}

\(\sigma\) is the Stefan-Boltzman constant.

The amount of energy dissipation in the canopy (leaf surface) is ,

\begin{eqnarray}
  \Delta F_c =
  H_c + R^{net}_c + l ( Et_{(1,2)} + Et_{(1,3)} )
  + l_s ( Et_{(2,2)} + Et_{(2,3)} )
\end{eqnarray}

However, \(R^{net}_c\) is the net radiative divergence in the canopy,

\begin{eqnarray}
  R^{net}_c = -(R^{\downarrow}_S - R^{\uparrow}_S) (1-{\mathcal{T}}_{cS})
              - \epsilon R^{\downarrow}_L (1-{\mathcal{T}}_{cL})
              + ( 2 \epsilon \sigma T_c^4
              - \epsilon \sigma T_s^4 ) (1 - {\mathcal{T}}_{cL})
\end{eqnarray}

\hypertarget{case-1-in-the-absence-of-surface-melting}{%
\subsubsection{Case 1: In the absence of surface
melting}\label{case-1-in-the-absence-of-surface-melting}}

In the absence of melting of the ground surface, as
\(\Delta F_s=\Delta F_c=0\), we solve for \(T_s\) and \(T_c\) so that
the energy balance between the ground surface and the canopy is
maintained.

The linearized energy balance equation for each term for \(T_s\) and
\(T_c\) is,

\begin{eqnarray}
 \left(
\begin{array}{l}
 \Delta F_s \\
 \Delta F_c \\
\end{array}
\right)^{current}
=
\left(
\begin{array}{l}
 \Delta F_s \\
 \Delta F_c \\
\end{array}
\right)^{past}
+
\left(
\begin{array}{ll}
 {\partial \Delta F_s}/{\partial T_s} 
 {\partial \Delta F_s}/{\partial T_c} \\
 {\partial \Delta F_c}/{\partial T_s} 
 {\partial \Delta F_c}/{\partial T_c} \\
\end{array}
\right)
\left(
\begin{array}{l}
 \Delta T_s \\
 \Delta T_c \\
\end{array}
\right)
=
\left(
\begin{array}{l}
 0 \\
 0 \\
\end{array}
\right)
\end{eqnarray}

I could write.

The part marked with \(past\) on the right-hand side is the flux
calculated from (107) to (134) using the values of \(T_s\) and \(T_c\)
in the previous step and substituted into (136) to (139).

The differential term is ,

\begin{eqnarray}
 \frac{\partial \Delta F_s}{\partial T_s} =
 \frac{\partial H_s}{\partial T_s}
+\frac{\partial R^{net}_s}{\partial T_s}
+l\frac{\partial Et_{(1,1)}}{\partial T_s}
+l_s\left(\frac{\partial Et_{(2,1)}}{\partial T_s}
+    \frac{\partial E_{Sn}}{\partial T_s}\right)
-\frac{\partial F_{g(1/2)}}{\partial T_s}
-\frac{\partial F_{Sn(1/2)}}{\partial T_s} \\
 \frac{\partial \Delta F_s}{\partial T_c} =
 \frac{\partial R^{net}_s}{\partial T_c} \\
 \frac{\partial \Delta F_c}{\partial T_s} =
 \frac{\partial R^{net}_c}{\partial T_s} \\
 \frac{\partial \Delta F_c}{\partial T_c} =
 \frac{\partial H_c}{\partial T_c}
+\frac{\partial R^{net}_c}{\partial T_c}
+l  \left(\frac{\partial Et_{(1,2)}}{\partial T_c}
+         \frac{\partial Et_{(1,3)}}{\partial T_c}\right)
+l_s\left(\frac{\partial Et_{(2,2)}}{\partial T_c}
+         \frac{\partial Et_{(2,3)}}{\partial T_c}\right)
\end{eqnarray}

However,

\begin{eqnarray}
 \frac{\partial R^{net}_s}{\partial T_s} =
 \epsilon 4 \sigma T_s^3 \\
 \frac{\partial R^{net}_s}{\partial T_c} =
 - ( 1 - {\mathcal{T}}_{cL} ) \epsilon 4 \sigma T_c^3 \\
 \frac{\partial R^{net}_c}{\partial T_s} =
 - ( 1 - {\mathcal{T}}_{cL} ) \epsilon 4 \sigma T_s^3 \\
 \frac{\partial R^{net}_c}{\partial T_c} =
  2( 1 - {\mathcal{T}}_{cL} ) \epsilon 4 \sigma T_c^3
\end{eqnarray}

Using the above, solve (140) for \(T_s\) and \(T_c\).

\hypertarget{case-2-when-there-is-ground-surface-melting}{%
\subsubsection{Case 2: When there is ground surface
melting}\label{case-2-when-there-is-ground-surface-melting}}

Melting of the ground surface occurs when there is snow or ice cover on
the ground surface and the temperature of the ground surface
(\(T_s^{current} = T_s^{past}+\Delta T_s\)) is higher than 0
\(^{\circ}\) C after the solution in Case 1. If there is ground surface
melting, the surface temperature is fixed at 0 \(^{\circ}\) C. In other
words, the surface temperature is fixed at 0 \(^{\circ}\) C,

\begin{eqnarray}
 \Delta T_s = \Delta T_s^{melt} = T_{melt} - T_s^{past}
\end{eqnarray}

is the melting point of ice (0 \(^{\circ}\) C). \(T_{melt}\) is the
melting point of ice (0 \(^{\circ}\) C).

Assuming that \(T_c\) is known, \(\Delta T_s\) is obtained by the
following equation as well as (140).

\begin{eqnarray}
 \Delta T_c = \left( - \Delta F_c^{past}
            - \frac{\partial \Delta F_c}{\partial T_s} \Delta T_s^{melt}
              \right) \Bigm/ \frac{\partial \Delta F_c}{\partial T_c}
\end{eqnarray}

If the \(\Delta T_s\) and \(\Delta T_c\) are thus known, the energy
convergence at the surface used for melting will be

\begin{eqnarray}
 \Delta F_{conv} =
 - \Delta F_s^{current} = - \Delta F_s^{past}
 - \frac{\partial \Delta F_s}{\partial T_s} \Delta T_s^{melt}
 - \frac{\partial \Delta F_s}{\partial T_c} \Delta T_c
\end{eqnarray}

It is obtained by.

\hypertarget{constraints-on-solutions.}{%
\subsubsection{Constraints on
solutions.}\label{constraints-on-solutions.}}

We set some constraints on the solution of the surface energy balance.
If the condition is violated, the energy balance is re-solved by fixing
the violated flux at the limit of the condition to be met.

\begin{enumerate}
\def\labelenumi{\arabic{enumi}.}
\tightlist
\item
  don't take too much water vapor from the first layer of the atmosphere
  A large downward latent heat may be generated due to temporary
  computational instability. Even in such a case, the condition is set
  so that all the water vapor from the surface to the first layer of the
  atmosphere is not lost.
\end{enumerate}

\begin{eqnarray}
  Et_{(i,j)}^{current} > - q_a ( P_s - P_a ) / (g \Delta t)
   \ \ \ \ \ (i=1,2 ; j=1,2,3) \\
  E_{Sn}^{current} > - q_a ( P_s - P_a ) / (g \Delta t)
\end{eqnarray}

Here, \(g\) and \(\Delta t\) are the acceleration due to gravity and the
time step of the atmospheric model. For the values such as \(Et\) used
in the judgment, an updated flux value (\(current\)) is used with
respect to the value of \(T_s\) and \(T_c\), which have been updated to
satisfy the energy balance. This is the same for all other cases below.
Updating the flux value will be described later.

\begin{enumerate}
\def\labelenumi{\arabic{enumi}.}
\setcounter{enumi}{1}
\tightlist
\item
  soil moisture is not negative. Prevent soil moisture from becoming
  negative through transpiration.
\end{enumerate}

\begin{eqnarray}
   Et_{(1,2)}^{current} <
     \sum_{k\in rootzone} \rho_w w_{k}\Delta z_{g(k)} /\Delta t_L
\end{eqnarray}

where \(\rho_w\) is the density of water and \(\Delta t_L\) is the time
step of the land surface model.

\begin{enumerate}
\def\labelenumi{\arabic{enumi}.}
\setcounter{enumi}{2}
\tightlist
\item
  no negative water content on the canopy Ensure that water on the
  canopy does not become negative by evaporation.
\end{enumerate}

\begin{eqnarray}
   Et_{(i,3)}^{current} < \rho_w w_c /\Delta t_L
   \ \ \ \ \ (i=1,2)
\end{eqnarray}

\begin{enumerate}
\def\labelenumi{\arabic{enumi}.}
\setcounter{enumi}{3}
\tightlist
\item
  the snowpack is not negative Ensure that the snowpack does not become
  negative due to sublimation of the snowpack.
\end{enumerate}

\begin{eqnarray}
   E_{Sn}^{current} < Sn /\Delta t_L
\end{eqnarray}

\hypertarget{ground-surface-canopy-temperature-update}{%
\subsubsection{Ground Surface, Canopy Temperature
Update}\label{ground-surface-canopy-temperature-update}}

Update surface and canopy temperatures.

\begin{eqnarray}
 T_s^{current} = T_s^{past} + \Delta T_s \\
 T_c^{current} = T_c^{past} + \Delta T_c
\end{eqnarray}

Based on the updated canopy temperature, it is necessary to diagnose
whether the water in the canopy is liquid or solid. This information
will be used in the future when dealing with the freeze and thaw of the
water on the canopy.

\begin{eqnarray}
 A_{Snc} = \left\{
\begin{array}{ll}
 0 (T_c \geq T_{melt})\\
 1 (T_c <    T_{melt})
\end{array}
\right.
\end{eqnarray}

The \(A_{Snc}\) is the freezing area fraction of water on the canopy.

\hypertarget{update-the-value-of-the-flux}{%
\subsubsection{Update the value of the
flux}\label{update-the-value-of-the-flux}}

Update the flux value with respect to the updated \(T_s\) and \(T_c\)
values. If you set the \(F\) to an arbitrary flux, updating the value is
done as follows

\begin{eqnarray}
 F^{current} = F^{past} + \frac{\partial F}{\partial T_s} \Delta T_s
                        + \frac{\partial F}{\partial T_c} \Delta T_c
\end{eqnarray}

The updated flux value is used to calculate the flux output to the
atmosphere.

\begin{eqnarray}
 H = H_s + H_c \\
 E = \sum_{j=1}^3 \sum_{i=1}^2 Et_{(i,j)} + E_{Sn} \\
 R^{\uparrow}_L = {\mathcal{T}}_{cL} \epsilon \sigma T_s^4
 + (1 - {\mathcal{T}}_{cL}) \epsilon \sigma T_c^4
 + (1 - \epsilon) R^{\downarrow}_L \\
 T_{sR} = ( R^{\uparrow}_L / \sigma )^{1/4}
\end{eqnarray}

\(T_{sR}\) is the radiant temperature of the ground surface.

The root uptake flux of each soil layer is calculated.

\begin{eqnarray}
 F_{root(k)} = f_{rootup(k)} Et_{(1,2)} \ \ \ \ (k=1,\ldots,K_g)
\end{eqnarray}

\(F_{root(k)}\) is the weight of the uptake flux of the roots, and
\(f_{rootup(k)}\) is the weight that distributes the transpiration rate
to the uptake flux of the roots in each layer.
