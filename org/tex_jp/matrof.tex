
\section{流出サブモデル MATROF}

簡略化した TOPMODEL (Beven and Kirkby, 1979) を用いて, 地表流出および地
下水流出を求める.

\subsection{TOPMODEL の概要}

TOPMODEL では, 流域内の斜面に沿った地下水面の水平分布を考える.
斜面のある点を下る地下水流は, その点より上方の斜面における地下水涵養量を
積算したものとつり合うと仮定する(準定常の仮定).
すると, 斜面の下方ほど地下水流が大きくなければならない. 後に述べる別の仮
定により, 地下水流が大きくなるためには地下水面が浅いことが必要とされる.
こうして, 斜面の下方ほど地下水面が浅いという分布が導出される.
平均的な地下水面がある程度より浅い場合, 斜面のある点から下では地下水面が
地面まで上がり, 飽和域を形成する.
このように, TOPMODEL では, 平均地下水面, 飽和域面積, 地下水流速という,
流出の見積もりにとって重要な概念が, 物理的に整合性をもって結びつく点に特
徴がある.

TOPMODEL では, 以下の3つの主要な仮定を置く.
\begin{enumerate}
  \item 土壌の飽和透水係数は, 土壌の深くに向かって指数関数的に減衰する.
  \item 地下水面の勾配は, 局所的には斜面の勾配とほぼ一致する.
  \item 斜面のある点を下る地下水流は, その点より上方の斜面における地下水
	涵養量を積算したものとつり合う.
\end{enumerate}

以下で, 記号の使い方は通常の TOPMODEL の記述の慣例に準じる
(Sivapalan et al., 1987 ; Stieglitz et al., 1997).

仮定1 は以下のように書ける.
\begin{equation}
 K_s(z) = K_0 \exp (-f z)
\end{equation}
$K_s(z)$ は深さ $z$ における土壌の飽和透水係数, $K_0$ は地表面における飽
和透水係数, $f$ は減衰係数である.

ある点 $i$ での地下水面の深さを $z_i$ とするとき, その点で斜面を下る地下
水フラックス $q_i$ は以下で表される.
\begin{equation}
 q_i = \int_{z_i}^Z K_s(z) dz \cdot \tan\beta
   = \frac{K_0}{f}  \tan\beta [\exp(-f z_i) - \exp(-f Z)]
 \label{top_qi}
\end{equation}
$\beta$ は斜面の勾配であり, ここで仮定2を用いた. $Z$ は不透水面の深さで
あるが, 通常 $Z$ は $1/f$ に比べて十分深いと仮定して, $\exp(-f Z)$
の項は省く. また, 地下水面より上の不飽和帯の斜面方向土壌水分フラックスは
小さいので無視する.

地下水涵養速度は水平一様に $R$ であるとすると, 仮定 3は以下のように表さ
れる.
\begin{equation}
 a R = \frac{K_0}{f} \tan\beta \exp(-f z_i)
\end{equation}
ここで, $a$ は, 地点 $i$ に対する上流総面積(地点 $i$ での単位等高線長さ
あたり)である.

これを $z_i$ について解くと, 以下を得る.
\begin{equation}
 z_i = -\frac{1}{f} \ln \left( \frac{faR}{K_0 \tan \beta}\right)
 \label{top_zi}
\end{equation}

領域 $A$ において平均した地下水面深さ $\overline{z}$ は,
\begin{equation}
   \overline{z} = \frac1{A}\int_{A} z_i dA
  = - \Lambda - \frac1{f} \ln R
 \label{top_zbar}
\end{equation}
\begin{equation}
 \Lambda \equiv
  \frac1{A}\int_{A} \ln \left( \frac{fa}{K_0 \tan \beta}\right) dA
\end{equation}

これにより, 涵養速度 $R$ が平均地下水面深さ $\overline{z}$ の関数として
以下のように表される.
\begin{equation}
 R = \exp (-f \overline{z} -\Lambda)
 \label{top_r}
\end{equation}
仮定 3により, これは領域 $A$ から排水される地下水流出量に他ならない.

次に, $R$ を (\ref{top_zi}) に代入すると以下の $z_i$ と $\overline{z}$
の関係を得る.
\begin{equation}
 z_i = \overline{z} - \frac{1}{f} \left[
\ln \left( \frac{fa}{K_0 \tan \beta}\right) - \Lambda
\right]
 \label{top_zizbar}
\end{equation}

$z_i \leq 0$ を満たす領域が地表飽和域である.

\subsection{簡略地形を仮定した TOPMODEL の適用}

通常, TOPMODEL が用いられる場合には, 対象地域の詳細な地形データを必要と
するが, ここではグリッドの平均的な傾斜と標高標準偏差のデータから, グリッ
ド内の斜面の平均的な形状を大まかに見積もる
(この見積もり方は現時点では暫定的なものであり, さらに検討を要する).

グリッド内の地形を, 一様な勾配$\beta_s$を持ち, 尾根から谷までの距離が
$L_s$ の斜面によって代表させる.

$L_s$ は, 標高標準偏差 $\sigma_z$ を用いて以下のように見積もる.
\begin{equation}
 L_s = 2\sqrt{3} \sigma_z / \tan\beta_s
\end{equation}
$2\sqrt{3}\sigma_z$ は, 標高標準偏差が $\sigma_z$ になるような鋸型の地形
における尾根と谷の標高差である.

水平面上に, 尾根から谷に向かって $x$軸を取る.
このとき, 地点 $x$ における上流総面積は $x$ であるから, (\ref{top_zi})
は以下のようになる.
\begin{equation}
 z(x) = - \frac{1}{f} \ln \left( \frac{fxR}{K_0 \tan \beta_s}\right)
\end{equation}
これを用いて, 平均地下水面は, (\ref{top_zbar})より
\begin{equation}
 \overline{z} = \frac 1{L_s}\int_0^{L_s} z(x) dx
 = - \frac1{f}\left[
 \ln \left( \frac{f L_s R}{K_0 \tan\beta_s}\right) -1
\right]
\end{equation}
地下水涵養速度は, (\ref{top_r}) より
\begin{equation}
 R = \frac{K_0 \tan\beta_s}{f L_s}\exp(1-f \overline{z})
\label{top_rb}
\end{equation}
地点 $x$ の地下水面と平均地下水面の関係は, (\ref{top_zizbar}) より
\begin{equation}
 z(x) = \overline{z} - \frac{1}{f}\left(
\ln \frac{x}{L_s} + 1
\right)
\end{equation}
となる.
$z(x) \leq 0$ を $x$ について解くと以下のようになる.
\begin{equation}
 x \geq x_0
\end{equation}
\begin{equation}
x_0 = L_s \exp(f\overline{z}-1)
 \end{equation}
従って, 飽和域の面積率は,
\begin{equation}
 A_{sat} = (L_s - x_0)/ L_s = 1 - \exp(f\overline{z}-1)
\label{top_asat}
\end{equation}
と求まる. ただし, $A_{sat} \geq 0$ であり, $\overline{z} > 1/f$ のときに
は飽和域は存在しない.

\subsection{流出量の計算}

4種類の流出メカニズムを考え, それぞれのメカニズムによる流出量の合計をグ
リッドからの総流出量とする.
\begin{equation}
 Ro = Ro_s + Ro_i + Ro_o + Ro_b
\end{equation}
$Ro_s$ は saturation excess runoff (Dunne runoff), $Ro_i$ は
infiltration excess runoff (Horton runoff), $Ro_o$ は土壌第1層のオーバー
フローであり, 以上は地表流出に分類される. $Ro_b$ は地下水流出である.

\subsubsection{平均地下水面深さの見積もり}

土壌水分量を土壌最下層から見ていき, 初めて不飽和になった層を $k_{WT}$ 層
目とするとき, 平均地下水面深さ $\overline{z}$を以下により見積もる.
\begin{equation}
 \overline{z} = z_{g(k_{WT}-1/2)} - \psi_{k_{WT}}
\label{roff_table}
\end{equation}
これは, 不飽和層の上端の水分ポテンシャルを$\psi_{k_{WT}}$と仮定し, その
下で土壌水分分布が平衡状態(重力と毛管力がつり合った状態)にあると考えてい
ることに相当する.

$\overline{z} > z_{g(k_{WT}+1/2)}$ のとき, $k_{WT}$ が最下層ならば地下水
面は存在しないとする. $k_{WT}$ が最下層でない場合には, ひとつ下の層(飽和
しているうちの最上層)を $k_{WT}$ として上式を適用する.

土壌の途中に凍土面がある場合は, 地下水面深さの見積もりは凍土面の上から行
う.

\subsubsection{地下水流出の計算}

地下水流出は, 準定常の仮定により (\ref{top_rb}) の地下水涵養速度と等しいので,
\begin{equation}
 Ro_b = \frac{K_0 \tan\beta_s}{f L_s}\exp(1-f \overline{z})
\end{equation}
である.
ただし, 地下水面の下に凍土面がある場合には, (\ref{top_qi}) の
$\exp(-fZ)$ の項を省かない場合を参考に,
\begin{equation}
 Ro_b = \frac{K_0 \tan\beta_s}{f L_s}
  [ \exp(1-f \overline{z}) - \exp(1-f z_f) ]
\end{equation}
とする. $z_f$ は凍土面の深さである.
このようにする場合, TOPMODEL の他の関係式も変わってくるはずであるが, 簡
略化のために他の関係式は変更しない.

また, 凍土面の下に不凍層があり, そこに地下水面が存在した場合には, そこか
らの地下水流出も同様に計算して加える.

地下水流出した水分は, 後で土壌第 $k_{WT}$ 層目から取り除く.
\begin{equation}
 Ro_{(k_{WT})} = Ro_b
\end{equation}
$Ro_{(k)}$ は第 $k$ 層目の土壌からの流出フラックスを表す.


\subsubsection{地表流出の計算}

地表飽和域に降った降水は全てそのまま流出する(saturation excess runoff).
\begin{equation}
 Ro_s = (Pr_c^{**} + Pr_l^{**}) A_{sat}
\end{equation}
地表飽和域の面積率 $A_{sat}$ は, (\ref{top_asat}) により与えられる.
ここで, サブグリッドの降水分布と地形との相関は無視している.

地表不飽和域に降った降水は, 土壌の浸透能を上回った分だけ流出する
(infiltration excess runoff).
土壌の浸透能は, 簡略化のため, 土壌第1層の飽和透水係数で与える.
対流性の降水は局所的に降ると考え, その降水域の面積率 $A_c$は一律に
仮定する(標準では 0.1). 層状性降水は一様と仮定する.
\begin{eqnarray}
 Ro_i^c &=& \max( Pr_c^{**}/A_c + Pr_l^{**} - K_{s(1)}, 0 ) (1 - A_{sat}) \\
 Ro_i^{nc} &=& \max( Pr_l^{**} - K_{s(1)}, 0 ) (1 - A_{sat})
\end{eqnarray}
\begin{equation}
 Ro_i = A_c Ro_i^c + ( 1 - A_c ) Ro_i^{nc}
\end{equation}
$Ro_i^c$, $Ro_i^{nc}$ は, それぞれ対流性降水域とそうでない場所の $Ro_i$
であり, $K_{s(1)}$ は土壌第1層の飽和透水係数である.

土壌第1層のオーバーフローは, わずかな湛水$w_{str}$(標準では 1mm)を許し
て,
\begin{equation}
 Ro_o = \max(w_{(1)} - w_{sat(1)} - w_{str}, 0) \rho_w \Delta z_{g(1)} / \Delta t_L
\end{equation}
とする. この分は後で土壌第1層から差し引くため, 第1層目からの流出量とし
て覚えておく.
\begin{equation}
 Ro_{(1)} = Ro_{(1)} + Ro_o
\end{equation}

\subsection{土壌へ与えられる水フラックス}

流出過程を経て土壌へ与えられる水フラックスは以下である.
\begin{equation}
 Pr^{***} = Pr^{**}_c + Pr^{**}_l - Ro_s - Ro_i
\end{equation}
