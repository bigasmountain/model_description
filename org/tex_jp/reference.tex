\subsection*{参考文献}

\begin{description}
 \item[] Ball, J. T., 1988: An analysis of stomatal
	    conductance. Ph.D. thesis, Stanford University, 89 pp.
 \item[] Beven, K. J., and M. J. Kirkby, 1979: A physically based
	    variable contributing area model of basin hydrology,
	    {Hydrol. Sci. Bull.}, {\bf 24}, 43--69.
 \item[] Clapp, R. B., and G. M. Hornberger, 1978: Empirical equations
	    for some soil hydraulic properties. {Water Resour. Res.},
	    {\bf 14}, 601--604.
 \item[] Collatz, G. J., J. A. Berry, G. D. Farquhar, and J. Pierce,
	    1990: The relationship between the Rubisco reaction
	    mechanism and models of leaf photosynthesis. {Plant Cell
	    Environ.}, {\bf 13}, 219--225.
 \item[] Collatz, G. J., J. T. Ball, C. Grivet, and J. A. Berry, 1991:
	    Physiological and environmental regulation of stomatal
	    conductance, photosynthesis and transpiration: A model that
	    includes a laminar boundary layer. {Agric. For. Meteor.},
	    {\bf 54}, 107--136.
 \item[] Collatz, G. J., M. Ribas-Carbo, and J. A. Berry, 1992: Coupled
	    Photosynthesis-Stomatal Conductance Model for leaves of
	    C$_4$ plants. {Aust. J. Plant. Physiol.}, {\bf 19},
	    519--538.
 \item[] Farquhar, G. D., S. von Caemmerer, and J. A. Berry, 1980: A
	    biochemical model of photosynthetic CO$_2$ fixation in
	    leaves of C$_3$ species. {Planta}, {\bf 149}, 78--90.
 \item[] Kondo, J., and T. Watanabe, 1992: Studies on the bulk transfer
	    coefficients over a vegetated surface with a multilayer
	    energy budget model. {J. Atmos. Sci}, {\bf 49}, 2183--2199.
 \item[] Rutter, B., A. J. Morton, and P. C. Robins, 1975: A predictive
	    model of rainfall interception in forests. II.
	    Generalization of the model and comparison with observations
	    in some coniferous and hardwood stands. {J. Appl. Ecol.},
	    {\bf 12}, 367--380.
 \item[] Sellers, P. J., D. A. Randall, G. J. Collatz, J. A. Berry,
	    C. B. Field, D. A. Dazlich, C. Zhang, G. D. Collelo, and
	    L. Bounoua, 1996: A revised land surface parameterization
	    (SiB2) for atmospheric GCMs. Part I: Model formulation.
	    {J. Climate}, {\bf 9}, 676--705.
 \item[] Sivapalan, M., K. Beven, and E. F. Wood, 1987: On hydrologic
	    similarity. 2, A scaled model of storm runoff
	    production. {Water Resour. Res}, {\bf 23}, 2266--2278.
 \item[] Stieglitz, M., D. Rind, J. Famiglietti, and C. Rosenzweig,
	    1997: An efficient approach to modeling the topographic
	    control of surface hydrology for regional and global climate
	    modeling. {J. Climate}, {\bf 10}, 118--137.
 \item[] Watanabe, T., 1994: Bulk parameterization for a vegetated
	    surface and its application to a simulation of nocturnal
	    drainage flow. {Boundary-Layer Met.}, {\bf 70}, 13--35.
 \item[] Wiscombe, W. J., and S. G. Warren, 1980: A model for the
	    spectral albedo of snow. I. Pure snow. {J. Atmos. Sci.},
	    {\bf 37}, 2712--2733.
 \item[] 渡辺力・大谷義一, 1995: キャノピー層内の日射量分布の近似計算法.
	    {農業気象}, {\bf 51}, 57--60.
\end{description}
